% Options for packages loaded elsewhere
\PassOptionsToPackage{unicode}{hyperref}
\PassOptionsToPackage{hyphens}{url}
%
\documentclass[
]{article}
\title{Summary Chapter 5}
\author{Christian}
\date{2023-10-21}

\usepackage{amsmath,amssymb}
\usepackage{lmodern}
\usepackage{iftex}
\ifPDFTeX
  \usepackage[T1]{fontenc}
  \usepackage[utf8]{inputenc}
  \usepackage{textcomp} % provide euro and other symbols
\else % if luatex or xetex
  \usepackage{unicode-math}
  \defaultfontfeatures{Scale=MatchLowercase}
  \defaultfontfeatures[\rmfamily]{Ligatures=TeX,Scale=1}
\fi
% Use upquote if available, for straight quotes in verbatim environments
\IfFileExists{upquote.sty}{\usepackage{upquote}}{}
\IfFileExists{microtype.sty}{% use microtype if available
  \usepackage[]{microtype}
  \UseMicrotypeSet[protrusion]{basicmath} % disable protrusion for tt fonts
}{}
\makeatletter
\@ifundefined{KOMAClassName}{% if non-KOMA class
  \IfFileExists{parskip.sty}{%
    \usepackage{parskip}
  }{% else
    \setlength{\parindent}{0pt}
    \setlength{\parskip}{6pt plus 2pt minus 1pt}}
}{% if KOMA class
  \KOMAoptions{parskip=half}}
\makeatother
\usepackage{xcolor}
\IfFileExists{xurl.sty}{\usepackage{xurl}}{} % add URL line breaks if available
\IfFileExists{bookmark.sty}{\usepackage{bookmark}}{\usepackage{hyperref}}
\hypersetup{
  pdftitle={Summary Chapter 5},
  pdfauthor={Christian},
  hidelinks,
  pdfcreator={LaTeX via pandoc}}
\urlstyle{same} % disable monospaced font for URLs
\usepackage[margin=1in]{geometry}
\usepackage{graphicx}
\makeatletter
\def\maxwidth{\ifdim\Gin@nat@width>\linewidth\linewidth\else\Gin@nat@width\fi}
\def\maxheight{\ifdim\Gin@nat@height>\textheight\textheight\else\Gin@nat@height\fi}
\makeatother
% Scale images if necessary, so that they will not overflow the page
% margins by default, and it is still possible to overwrite the defaults
% using explicit options in \includegraphics[width, height, ...]{}
\setkeys{Gin}{width=\maxwidth,height=\maxheight,keepaspectratio}
% Set default figure placement to htbp
\makeatletter
\def\fps@figure{htbp}
\makeatother
\setlength{\emergencystretch}{3em} % prevent overfull lines
\providecommand{\tightlist}{%
  \setlength{\itemsep}{0pt}\setlength{\parskip}{0pt}}
\setcounter{secnumdepth}{-\maxdimen} % remove section numbering
\ifLuaTeX
  \usepackage{selnolig}  % disable illegal ligatures
\fi

\begin{document}
\maketitle

\hypertarget{what-are-assumptions}{%
\subsection{1. What are assumptions?}\label{what-are-assumptions}}

In statistical analysis, all parametric tests rely on specific data
characteristics, which, if not met, can alter the research findings and
the way results are interpreted.The mandatory condition that the data
meet this requirement constitutes an assumption.

\hypertarget{assumptions-of-parametric-data}{%
\subsection{2. Assumptions of parametric
data}\label{assumptions-of-parametric-data}}

A parametric test is one that requires data from one of the large
catalogue of distributions that statisticians have described, and for
data to be parametric certain assumptions must be true:

\begin{itemize}
\item
  \textbf{normally distributed data} (e.g.~symmetry, unimodality,
  bell-shaped form, AUC=1)
\item
  \textbf{homogenity of variance} (dispersion of data points around
  regression line or mean value is consistent)
\item
  \textbf{interval data}
\item
  \textbf{independence} (between-subject design: each participants data
  does not influence other participants; within subject design: no
  independent data within one subject, but between subjects)
\end{itemize}

\hypertarget{the-assumption-of-normality}{%
\subsection{3. The assumption of
normality}\label{the-assumption-of-normality}}

Prerequisities of the distribution differs depending on the the
statistical tests. For example, if we need a bell-shaped normal
distribution (e.g.~for a t-test), the larger the data set, the more
likely it is to be present (at least n=30, central limit theorem). For
GLM (General Linear Models, e.g.~regression) we have to assume that the
errors are distributed normally.

\hypertarget{visual-inspection}{%
\subsubsection{3.1 Visual inspection}\label{visual-inspection}}

To define if it is a normal distribution, we can visually inspect the
data in a frequency distribution. We can create the distribution with
the function \textbf{ggplot2.}

The ggplot2 tool offers the possibility to plot a line over the existing
distribution that describes a normal distribution. Based on this we can
decide if the data meets the requirements (visually).

Furthermore, we can create a \textbf{Q-Q plot}. This graph plots the
cumulative values we have in our data against the cumulative probability
of a particular distribution. If the data are normally distributed, then
the observed values (dots on the chart) should fall exactly along a
straight line. Any deviation of the dots from the line represents a
deviation from normality.

If our analysis involves \textbf{comparing groups}, then what's
important is not the overall distribution but the distribution in each
group. In this case, we can still use the \textbf{ggplot()} function,
but we need to define a factor variable defining subsampling. Therefore
we can use the function \textbf{subset()} and apply it to ggplot().
After the subsampling is applied (a new dataframe is being created) as
follows, we can use our ``normal'' ggplot() command.

\hypertarget{subsample-subsetdf-dffactorvariablefactor-characteristic}{%
\subsubsection{subsample\textless-subset(df,
df\$FactorVariable==``Factor
Characteristic'')}\label{subsample-subsetdf-dffactorvariablefactor-characteristic}}

\hypertarget{quantifying-normality-with-numbers}{%
\subsubsection{3.2 Quantifying normality with
numbers}\label{quantifying-normality-with-numbers}}

To quantify if our data fit a normal distribution, we can use the
functions stat.desc() and describe() from the psych package.

\hypertarget{stat.descvariable-name-basic-true-norm-false}{%
\subsubsection{stat.desc(variable name, basic = TRUE, norm =
FALSE)}\label{stat.descvariable-name-basic-true-norm-false}}

\hypertarget{describedfvariable}{%
\subsubsection{describe(df\$variable)}\label{describedfvariable}}

The function gives a multiple output. Most important for checking the
normal distribution are the \textbf{standardized values} of skew and
kurtosis, and the result of the Shapiro-Wilk test.

\begin{itemize}
\item
  skew: indicates how asymmetric or skewed the distribution is

  \begin{itemize}
  \item
    \textgreater0: too many low scores in the distribution (``linkssteil
    und rechtsschief'')
  \item
    \textless0: too many high scores in the distribution ``linksschief
    und rechtssteil'')

    \begin{itemize}
    \tightlist
    \item
      visual observation in QQ-plots: S-shaped curve not fitting the
      line -\textgreater{} different skewness to normal distribution
    \end{itemize}
  \end{itemize}
\item
  kurtosis: measures the shape of the distribution with respect to its
  edges and whether the maximum is flat or pointed

  \begin{itemize}
  \item
    \textgreater0: pointy and heavy tailed distribution
  \item
    \textless0: flat and light tailed distribution

    \begin{itemize}
    \tightlist
    \item
      visual observation in QQ-plots:: data consistently above or under
      the line -\textgreater{} different curtosis to normal distribution
    \end{itemize}
  \end{itemize}
\item
  \textbf{shapiro.test(variable)}

  \begin{itemize}
  \tightlist
  \item
    Shapiro-Wilk test: p-values significant -\textgreater{} no normal
    distribution
  \end{itemize}
\end{itemize}

If we are analyzing \textbf{different and independent groups} it is
important to test each groups distribution for normality individually.
If we tested the overall sample, group differences could lead to
distortions like bimodal distributions.

Therefore we can use the function by() that is then connected to
describe() or stat.desc()

\hypertarget{bydata-dataframe-indices-grouping-variable-fun-describe}{%
\subsubsection{by(data = dataFrame, INDICES = grouping variable, FUN =
describe)}\label{bydata-dataframe-indices-grouping-variable-fun-describe}}

\hypertarget{bydata-dataframe-indices-grouping-variable-fun-stat.desc}{%
\subsubsection{by(data = dataFrame, INDICES = grouping variable, FUN =
stat.desc)}\label{bydata-dataframe-indices-grouping-variable-fun-stat.desc}}

\hypertarget{testing-for-homogenecity-of-variance}{%
\subsection{4. Testing for homogenecity of
variance}\label{testing-for-homogenecity-of-variance}}

Through all levels of a variable, the variance should not change.

\begin{itemize}
\item
  For grouped data, the assumption is that the variance of the outcome
  variable(s) should be consistent across all groups.
\item
  For continuous data, such as in correlational designs, the assumption
  is that the variance of one variable should remain stable at all
  levels of the other variable.
\end{itemize}

\hypertarget{levences-test-for-comparing-group-variance}{%
\subsubsection{4.1 Levence´s test for comparing group
variance}\label{levences-test-for-comparing-group-variance}}

Visual testing of variance homogenecity is vulnerable to error.
Therefore one should use objective mathematical tests, as the Levence´s
test. This is basically a one-way ANOVA conducted on the deviation
scores. If the test is significant, the assumption is violated. If
Levene's test is significant (Pr (\textgreater F) in the R output is
less than .05) then the variances are significantly different in
different groups.

In R, the test can be performed with the R-Commander, see p.200 or
normally, via the console, included in the car package. Here, we need to
define the outcome variable and a factor/group variable.

\hypertarget{levenetestoutcome-variable-group-center-medianmean}{%
\paragraph{leveneTest(outcome variable, group, center =
median/mean)}\label{levenetestoutcome-variable-group-center-medianmean}}

Warning: In large samples Levene's test can be significant even when
group variances are not very different. Therefore, it should be
interpreted in conjunction with the variance ratio.

\hypertarget{hartleys-fmax-the-variance-ratio}{%
\subsubsection{4.2 Hartley´s Fmax: the variance
ratio}\label{hartleys-fmax-the-variance-ratio}}

If the sample is too large, tests as described before can be
significant, even though variance homogenecity exists. Hartley´s Fmax
tests the variance ratio, which is the largest group variance divided by
the smallest. The critical values defining if the assumption is
fulfilled depend on the degrees of freedom (df)

\hypertarget{correcting-problems-in-the-data}{%
\subsection{5. Correcting problems in the
data}\label{correcting-problems-in-the-data}}

\hypertarget{outliers}{%
\subsubsection{5.1 Outliers}\label{outliers}}

Options to deal with detected outliers:

\begin{itemize}
\item
  remove the cases - only if a good reason exists!
\item
  transform the data
\item
  change the score - if transformation fails

  \begin{itemize}
  \item
    change the score to be one unit above the next highest score in the
    dataset
  \item
    convert back from a z-score
  \item
    the mean plus two standard deviations
  \end{itemize}
\end{itemize}

\hypertarget{dealing-with-non-normality-and-unequal-variances}{%
\subsubsection{5.2 Dealing with non-normality and unequal
variances}\label{dealing-with-non-normality-and-unequal-variances}}

\hypertarget{transforming-data}{%
\paragraph{Transforming data}\label{transforming-data}}

The idea behind transformations is that you do something to every score
to correct for distributional problems, outliers or unequal variances.
This does not change the relations between variables.

Careful: If you are comparing groups (or different datasets in general),
every subset of data has to be transformed that comparability is still
guaranteed!

\hypertarget{choosing-a-transformation}{%
\paragraph{Choosing a transformation}\label{choosing-a-transformation}}

trial and error!

\hypertarget{transforming-the-data-using}{%
\subsubsection{5.3 Transforming the data
using}\label{transforming-the-data-using}}

\hypertarget{computing-new-variables}{%
\paragraph{computing new variables}\label{computing-new-variables}}

only two general commands are required in R:

\begin{itemize}
\item
  newVariable \textless- function(oldVariable)

  \begin{itemize}
  \item
    log() - if there are variables with the value 0, add +1
  \item
    sqrt() - square root transformation
  \item
    1/() - reciprocal transformation (more like an arithmetic
    expression)
  \item
    ifelse(a conditional argument, what happens if the argument is TRUE,
    what happens if the argument if FALSE) - creates new variable
    containing all cases for which argument is true
  \end{itemize}
\item
  newVarielse() - able \textless- arithmetic with oldVariable(s)

  \begin{itemize}
  \item
    Addition +
  \item
    Subtraction -
  \item
    Exponentation ** OR \^{}x
  \item
    Less than \textless; greater than \textgreater; less than or equal
    to \textless=; greater than or equal to \textgreater=
  \item
    double equals ==
  \item
    not equal to !=
  \end{itemize}
\end{itemize}

\hypertarget{the-effect-of-transformations}{%
\subsubsection{The effect of
transformations}\label{the-effect-of-transformations}}

Mainly, transformations reduce the expression of the violation of the
assumptions. When transforming distributions that are already normally
distributed, it can lead to slight violations of the assumptions.

Therefore, it is important to find a trade-off to decide whether it
makes sense to perform a transformation. This is mostly a
trial-and-error process.

\hypertarget{when-it-all-goes-horribly-wrong}{%
\subsubsection{5.4 When it all goes horribly
wrong}\label{when-it-all-goes-horribly-wrong}}

If transformation does not work, parametric tests can be used.

Otherwise, so-called robust methods can be used. Some of these
procedures use a trimmed mean. A trimmed mean is simply a mean based on
the distribution of scores after some percentage of scores has been
removed from each extreme of the distribution.

A similar robust measure of location is an M-estimator, which differs
from a trimmed mean in that the amount of trimming is determined
empirically. In other words, rather than the researcher deciding before
the analysis how much of the data to trim, an Mestimator determines the
optimal amount of trimming necessary to give a robust estimate of, say,
the mean.

Furthermore, subsampling methods such as cross-validation and
bootstrapping can be used. Here, randomized samples are taken from the
existing data set and thus a new distribution with more values is
created. In this way, larger distributions can be simulated, thus
meeting the requirements.

\end{document}
